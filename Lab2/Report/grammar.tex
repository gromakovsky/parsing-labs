\section{Разработка грамматики}

Описания переменных в Си. Сначала следует имя типа, затем разделенные запятой имена переменных. Переменная может быть указателем, в этом случае перед ней идет звездочка (возможны и указатели на указатели, и т. д.). Описаний может быть несколько.

\subsection{Грамматика}

\begin{center}
    \begin{tabular}{ r c l }
        $S$ & $\rightarrow$ & $\mathrm{n} VN\mathrm{;} S^{'}$ \\
        $S^{'}$ & $\rightarrow$ & $S$ \\
        $S^{'}$ & $\rightarrow$ & $\varepsilon$ \\
        $V$ & $\rightarrow$ & $\mathrm{*} V$ \\
        $V$ & $\rightarrow$ & $\mathrm{n}$ \\
        $N$ & $\rightarrow$ & $\mathrm{,} VN$ \\
        $N$ & $\rightarrow$ & $\varepsilon$ \\
    \end{tabular}
\end{center}

\subsection{Описания нетерминалов}

\begin{center}
    \begin{tabular}{ | l | l | }
        \hline
        \textbf{Нетерминал} & \textbf{Описание} \\
        \hline
        $S$ & Первое описание переменных. \\
        \hline
        $S^{'}$ & Следующее описание. \\
        \hline
        $V$ & Описание одной переменной. \\
        \hline
        $N$ & Описание следующей переменной. \\
        \hline
    \end{tabular}
\end{center}

\subsection{Удаление левой рекурсии и правого ветвления}
В получившейся грамматике нет ни левой рекурсии, ни правого ветвления.
